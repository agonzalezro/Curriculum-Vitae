\documentclass[11pt, a4paper, sans]{moderncv}

\linespread{1.1}

\usepackage{url}

% colors: black, blue, green, grey, orange & purple
% styles: banking, casual, classic & oldstyle
\moderncvtheme[grey]{banking}

% character encoding
\usepackage[utf8]{inputenc}
\usepackage[english]{babel}
\usepackage{datetime}

\newcommand{\revision}{\twodigit\day\twodigit\month\the\year}

\usepackage{fancyhdr}
\pagestyle{fancy}
\fancyfoot[RE, RO]{\footnotesize{rev. \revision}}

% adjust the page margins
\usepackage[scale=.85]{geometry}
\setlength{\hintscolumnwidth}{3cm}
\AtBeginDocument{\recomputelengths}

% personal data
\firstname{Álex}
\familyname{González}

\phone[mobile]{\texttt{+34 640 265 101}}
\homepage{http://agonzalezro.github.io}
\email{agonzalezro@gmail.com}

\social[github]{agonzalezro}
\social[twitter]{agonzalezro}
\social[linkedin]{alexandregonzalezrodriguez}

\begin{document}
\makecvtitle


%
% PROFESSIONAL EXPERIENCE
%

One of the parts I like the most about our field is its community scene, for that reason apart of my day to day job
I spend some time organizing events such as \href{https://www.meetup.com/mad-scalability/}{madScalability} or
the first \href{https://golanguk.com/}{Golang UK conference} as well as speaking in them,
for example, I did that at \href{https://www.kubecon.io}{Kubecon}, \href{https://www.gophercon.co.uk/}{Gophercon} or \href{https://codemotionworld.com/}{Codemotion}.\\
\\
I also have some FLOSS projects that can be checked on \href{https://github.com/agonzalezro}{my GitHub profile}.
To be fair, the most interesting work I did is mainly closed source but I keep myself updated doing some small contributions to interesting projects as \href{https://github.com/agonzalezro/cri-o}{cri-o}, \href{https://github.com/containers/image}{image} and others.
When I had the opportunity of working professionally in the OpenSource space I did the Flocker plugin for Kubernetes: \href{https://github.com/kubernetes/kubernetes/pull/14328}{pr\#14328}.\\

\subsection{Professional experience}
\cventry{Nov'16--Cur.}
{BBVA Next (BBVA group)}
{Head of Backend @ BBVA Next \& Team Lead @ BBVA CaaS}{Spain}{}
{Working on a Containers As A Service project for the BBVA core banking processes.\\
Our main concern as a team is to comply and enforce the different SLAs defined in our platform. Since our colleagues deploy high sensitive applications that move and manage money, the Platform not only has to work, it has to provide enough feedback for different teams to successfully run their applications.\\
As a Team Lead, I’m in constant contact with the Product team. I work on technical definitions. I make sure that the code committed follows the high standards defined by business, manage and mentor fellow developers and I try to be as accessible as possible to the other business units.\\
On the other hand, as Head of Backend in Next I am more related with managment tasks such as writing down the technical path for the company using assets as the Radar, preparation of interviews, collaborate on the promotions process or representing the company during introductions to possible collaborators.\\}

\cventry{Oct'15--Oct'16}
{Jobandtalent}{Software architect}{Spain}{}
{Working towards a migration to Docker services from development with
  Docker Compose and internal tools (Go \& Ruby) all the way up to
  production with ECS. Terraform and Packer were also involved here.\\
  Started the migration to terraform infrastructure.\\
Development of a geolocation service that consumes MaxMind DB, GMaps and
internal data.\\
Planning of the bounded context of the services in the new revision of the
architecture and the new API guidelines.\\
Investigation of possible future technology choices as grpc, terraform, GKE\ldots\\}

\cventry{Sept'15--Oct 2015}
{Beyond Seeds Ltd}
{Contractor}{UK}{}
{I developed the Flocker plugin for Kubernetes (contracting with Jetstack). The resultant code is now part of the Kubernetes codebase.\\
For another contract I developed a Python API to expose a NumPy/SciPy function. I made the deployment into Kubernetes (GKE).\\}

\cventry{Oct'14--Sept 2015}
{Shopa}
{Developer}{UK}{}
{I was working with Go and Rails applications to build a social marketplace. The
organisation of this company was completely flat so the decisions about
architecture or technologies were delegated to us. I was as well working with
integration with some external companies.\\I was at the same time involved in
devops work related with Ansible, Docker and AWS services.\\}

\cventry{Apr'13--Oct 2014}
{Green Man Gaming}
{Engineering Manager}{UK}{}
{On this position I had the chance of see the projects growing from the
beginning. I was in charge of architect them with the team and keep track that
all the tasks were being done properly and later, merge them. I developed  as
well every time that I had the occasion and of course, I did peer reviews.\\I
was developing mainly with Python and Golang languages, using several DB
backends: CouchDB, Redis, MySQL, PostgreSQL and several frameworks as could be:
Flask, Django or Revel.\\}

\cventry{Dec'12--Apr 2013}
{Green Man Gaming}
{Software Engineer}{UK}{}
{I was working in the several services that the company used to provide the
shopping and social networking (we own playfire.com too) experience. Those
services were mainly writing in Python, Scala and NodeJS. I was giving support
in the CI and testing environment that we were growing at that time. Part of my
work was doing code-reviews too.\\}

\cventry{May'11--Nov 2012}
{Paylogic}
{Developer}{Netherlands}{}
{Backend developer creating a new RESTful API (Python, flask, PostgreSQL,
celery\ldots) that was the entry point between the mobile applications and all
the Paylogic services and social networks.\\Sometimes I was working directly in
Paylogic's backoffice where I had the chance to use Django, backbone and others.\\}

\cventry{Apr'10--Jan 2011}
{lukkom}
{Technology responsible}{Spain}{}
{Web platform and REST API development, management consulting and secondary IT
works. All this project was created from scratch by me and I've decided to go
with Python related technologies: Django, django-piston, celery\ldots\\}

\cventry{Mar'09--Apr 2010}
{mirblu S.L.}
{Founder}{Spain}{}
{Development in several languages: Python, PHP, Java, Bash Scripting and Visual
Basic. Customer search. Business administration works.\\The main project of the
Company was an indoor position system created with J2ME techonologies. This
project was selected as one of the best I+D projects of Castilla y León.\\}

\cventry{2004--2009}{}{Some other \textit{non-that-interesting} IT \& non-IT jobs}{Spain}{}{}

%
% EDUCATION
%

\subsection{Education}
\cventry{2009--2011}
{University of Salamanca}{MSc Electronic Commerce}{Spain}{}{}

\cventry{2004--2009}
{Pontifical University of Salamanca}{BSc Computer Science}{Spain}{}{}

\cventry{2002--2004}
{Tesdai}{Higher level vocational training at computer systems management}{Spain}{}{}

%
% LANGUAGES
%

\subsection{\\Languages}
\cvdoubleitem{\textbf{Fluent}}{English, Spanish \& Galician}{\textbf{Understanding level}}{Portuguese}
\cvitem{\textbf{Intermediate}}{Italian}
\end{document}
